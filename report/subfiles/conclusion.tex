\chapter{Conclusion}

% 1 paragraph summarizing what we did
We consider two tasks to solve which are the regression problem and the logistic regression. Linear regression helps us to predict the actual number of graffiti that will be on each building, meanwhile logistic regression helps us to predict whether a building will be vandalized with graffiti or not. We use the linear regression to highlight and explain the most contributing features, because it is much more interpretable than a Neural Network. Neural network is not drastically more accurate than the base logistic regression model as well.

% 1 paragraph saying what we learnt
We use the Linear Regression to highlight and explain the most contributing features, because it is much more interpretable than a Neural Network. From the deter graffiti features, we learnt that the higher population in an area, the less chance of graffiti. Furthermore, building on a residential street are less likely to be vandalized. Areas which are more dense have an advantage over sparser areas. Lastly, tall buildings in the vicinity of a building were found to also contribute towards the deterrence of graffiti. On the other side, from our likelihood of graffiti features we learnt that more graffiti  in the vicinity of a building will increase the chances of that building having graffiti. Buildings on arterial streets and those with flat roof types will also have a higher chance of being vandalized.


% 1 paragraph describing limitations and future work
Some datasets were not always up to date, for example, the buildings footprint dataset is from 2009. Our methodology does not take this into account, hence, rerunning the project with updated data may lead to more results more representative of the current situation.

% Given the same features, we would be able to do the same for different cities around the world.
Our result will be displayed in an online blog which will be aimed at officials working at the city council of Vancouver. The result will be suit perfectly for the Vancouver officials because we made a model based on the Vancouver dataset. We will also be able to do the same for different cities around the worlds if there’s similar features. However, other officials from all over the globe may also view it and adapt to their problem as well. Therefore, it may prompt a new way of dealing with graffiti before it occurs. Moreover, city planners may use this to design cities which are graffiti proof by design. Finally, to inspire other researchers to explore this area further, they can see our methodology and technical details in this report.