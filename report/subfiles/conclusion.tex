\chapter{Conclusion}

% 1 paragraph summarizing what we did
We consider two tasks to solve which are the regression problem and the classification. Regression problem helps us to predict the actual number of graffiti that will be on each building, meanwhile classification helps us to predict whether a bulding will be vandalized with graffiti or not.

% 1 paragraph saying what we learnt


% 1 paragraph describing limitations and future work
Some datasets were not always up to date, for example, the buildings footprint dataset is from 2009. Our methodology does not take this into account, hence, rerunning the project with updated data may lead to more results more representative of the current situation.

% Given the same features, we would be able to do the same for different cities around the world.
Our result will be displayed in an online blog which will be aimed at officials working at the city council of Vancouver.
However, other officials from all over the globe may also view it and adapt to their problem as well.
Therefore, it may prompt a new way of dealing with graffiti before it occurs.
Moreover, city planners may use this to design cities which are graffiti proof by design.
Lastly, to inspire other researchers to explore this area further, they can see our methodology and technical details in this report.